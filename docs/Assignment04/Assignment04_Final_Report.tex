\documentclass[12pt,a4paper]{article}
\usepackage[utf8]{inputenc}
\usepackage[margin=1in]{geometry}
\usepackage{graphicx}
\usepackage{float}
\usepackage{longtable}
\usepackage{booktabs}
\usepackage{array}
\usepackage{pdflscape}
\usepackage{multirow}
\usepackage{xcolor}
\usepackage{colortbl}
\usepackage{hyperref}
\usepackage{fancyhdr}
\usepackage{titlesec}
\usepackage{enumitem}
\usepackage{caption}

% Page setup
\pagestyle{fancy}
\fancyhf{}
\fancyhead[L]{Assignment 04 - Resource Management}
\fancyhead[R]{Mindful Eating Agent}
\fancyfoot[C]{\thepage}

% Title formatting
\titleformat{\section}{\Large\bfseries}{\thesection}{1em}{}
\titleformat{\subsection}{\large\bfseries}{\thesubsection}{1em}{}

% Hyperlink setup
\hypersetup{
    colorlinks=true,
    linkcolor=blue,
    filecolor=magenta,      
    urlcolor=cyan,
}

\begin{document}

% Cover Page
\begin{titlepage}
    \centering
    \vspace*{2cm}
    
    {\Huge\bfseries Assignment 04\par}
    \vspace{0.5cm}
    {\LARGE Resource Management and Leveling\par}
    \vspace{2cm}
    
    {\Large\bfseries Mindful Eating Agent Development Project\par}
    \vspace{1cm}
    
    {\large Course: Fundamentals of Software Project Management\par}
    {\large Section: E\par}
    \vspace{2cm}
    
    {\large\bfseries Team Members:\par}
    \vspace{0.5cm}
    \begin{tabular}{|c|c|c|}
        \hline
        \textbf{Name} & \textbf{Roll Number} & \textbf{Role} \\
        \hline
        Dawood Hussain & 22i-2410 & Project Manager \\
        \hline
        Gulsher Khan & 22i-2637 & Technical Lead \\
        \hline
        Ahsan Faraz & 22i-8791 & AI/ML Developer \\
        \hline
    \end{tabular}
    
    \vspace{2cm}
    {\large Submission Date: November 30, 2025\par}
    \vspace{1cm}
    
    {\large\bfseries Project Overview:\par}
    \vspace{0.3cm}
    \begin{tabular}{ll}
        Original Duration: & 112 days (Sep 1 - Nov 30, 2025) \\
        Updated Duration: & 112 days (Sep 1 - Nov 30, 2025) \\
        Extension: & No extension (deadline maintained) \\
    \end{tabular}
    
    \vspace{1cm}
    {\large\bfseries Tech Stack:\par}
    \vspace{0.3cm}
    \begin{itemize}[left=2cm]
        \item Backend: Flask (Python)
        \item Frontend: HTML/CSS (Flask templates)
        \item AI Framework: LangGraph
        \item Database: PostgreSQL
        \item Deployment: AWS
    \end{itemize}
    
\end{titlepage}

\newpage
\tableofcontents
\newpage

\section{Executive Summary}

This report presents a comprehensive resource management analysis for the Mindful Eating Agent project, focusing on resource assignment, loading analysis, leveling techniques, and schedule optimization under resource constraints.

\subsection{Key Findings}

\textbf{Initial Schedule Issues:}
\begin{itemize}
    \item \textbf{Gulsher Khan:} Over-allocated at 120\% (48 hours/week) during Weeks 8-13
    \item \textbf{Ahsan Faraz:} Over-allocated at 120\% (48 hours/week) during Weeks 8-13
    \item \textbf{Dawood Hussain:} Under-allocated at 60-70\% during development phase
\end{itemize}

\textbf{After Resource Leveling:}
\begin{itemize}
    \item All team members balanced at $\leq$100\% allocation
    \item Project duration: 112 days (maintained original deadline)
    \item Completion date: November 30, 2025
    \item Sustainable workload distribution achieved
    \item Quality risk reduced through adequate task time allocation
\end{itemize}

\subsection{Assignment Deliverables}

This assignment addresses four main tasks:

\begin{enumerate}
    \item \textbf{Resource Assignment Matrix (RAM)} - RACI matrix showing clear responsibility assignments for all WBS activities
    \item \textbf{Resource Loading Analysis} - Week-by-week workload distribution with identification of over/under-allocation
    \item \textbf{Resource Leveling} - Application of leveling techniques to resolve conflicts and balance workload
    \item \textbf{Updated Network Diagram \& Schedule} - Adjusted project timeline reflecting resource constraints
\end{enumerate}

\newpage

\section{Task 1: Resource Assignment Matrix (RAM)}

\subsection{Overview}

The Responsibility Assignment Matrix (RAM) uses the RACI model to clearly define roles and responsibilities for each project activity:

\begin{itemize}
    \item \textbf{R} = Responsible (Does the work)
    \item \textbf{A} = Accountable (Final authority/approval)
    \item \textbf{C} = Consulted (Provides input)
    \item \textbf{I} = Informed (Kept updated)
\end{itemize}

\subsection{Team Member Responsibilities}

\subsubsection{Dawood Hussain (Project Manager)}

\textbf{Primary Responsibilities (R):}
\begin{itemize}
    \item Project Coordination (119 days)
    \item Risk Management (119 days)
    \item Stakeholder Identification (5 days)
    \item Business Case Development (5 days)
    \item Requirements Gathering (6 days)
    \item Schedule Development (8 days)
    \item Risk Planning (7 days)
    \item User Acceptance Testing (5 days)
    \item User Training (2 days)
    \item All closure activities (4 days total)
\end{itemize}

\subsubsection{Gulsher Khan (Technical Lead)}

\textbf{Primary Responsibilities (R):}
\begin{itemize}
    \item Change Management (76 days)
    \item Feasibility Study (7 days)
    \item System Architecture Design (8 days)
    \item UI/UX Design (11 days)
    \item Environment Setup (10 days)
    \item Backend API Development - Flask (14 days)
    \item Frontend Development - HTML/CSS/Flask (20 days)
    \item Integration Testing (10 days)
    \item Production Environment Setup (4 days)
    \item Production Deployment (2 days)
    \item Knowledge Transfer (1 day)
\end{itemize}

\subsubsection{Ahsan Faraz (AI/ML Developer)}

\textbf{Primary Responsibilities (R):}
\begin{itemize}
    \item Market Research (5 days)
    \item Database Design (10 days)
    \item LangGraph Agent Development (24 days total):
    \begin{itemize}
        \item Agent Workflow Design (6 days)
        \item LangGraph Implementation (9 days)
        \item Agent Testing \& Validation (6 days)
        \item Agent Integration (3 days)
    \end{itemize}
    \item Functional Testing (6 days)
    \item Quality Management (shared with Dawood, 119 days)
\end{itemize}

\subsection{Collaborative Tasks}

Several critical tasks require collaboration across team members:

\begin{itemize}
    \item \textbf{Integration Testing:} Gulsher (R), Ahsan (A), Dawood (C)
    \item \textbf{Agent Integration:} Ahsan (R), Gulsher (A)
    \item \textbf{Quality Management:} Dawood (R), Ahsan (R), Gulsher (A)
    \item \textbf{Risk Planning:} Dawood (R), Gulsher (A), Ahsan (C)
    \item \textbf{Requirements Gathering:} Dawood (R), Ahsan (A), Gulsher (C)
\end{itemize}

\subsection{Complete RACI Matrix}

The complete Resource Assignment Matrix is provided in Table~\ref{tab:raci} on the following page, showing RACI assignments for all 40 project activities across 6 phases.

\newpage

\subsection{Resource Assignment Matrix (RAM)}

The following table presents a clean RACI matrix showing responsibility assignments:

\begin{table}[H]
\centering
\caption{Resource Assignment Matrix - Summary by Phase}
\label{tab:ram-summary}
\small
\begin{tabular}{|l|c|c|c|}
\hline
\textbf{Phase / Activity} & \textbf{Dawood} & \textbf{Gulsher} & \textbf{Ahsan} \\
\hline
\multicolumn{4}{|c|}{\textbf{Continuous Activities}} \\
\hline
Project Coordination & R & C & I \\
Risk Management & R & C & A \\
Change Management & C & R & I \\
Quality Management & R & A & R \\
\hline
\multicolumn{4}{|c|}{\textbf{Phase 1: Initiation}} \\
\hline
Market Research & C & I & R \\
Stakeholder Identification & R & C & I \\
Feasibility Study & I & R & C \\
Business Case Development & R & C & I \\
\hline
\multicolumn{4}{|c|}{\textbf{Phase 2: Planning}} \\
\hline
Requirements Gathering & R & C & A \\
System Architecture Design & C & R & I \\
UI/UX Design & I & R & C \\
Database Design & I & C & R \\
Schedule Development & R & C & I \\
Risk Planning & R & A & C \\
\hline
\multicolumn{4}{|c|}{\textbf{Phase 3: Development}} \\
\hline
Environment Setup & I & R & C \\
Backend API (Flask) & C & R & I \\
LangGraph Agent Development & I & C & R \\
Frontend (HTML/CSS/Flask) & C & R & I \\
Integration Testing & C & R & A \\
\hline
\multicolumn{4}{|c|}{\textbf{Phase 4: Testing \& Deployment}} \\
\hline
Functional Testing & C & I & R \\
User Acceptance Testing & R & C & I \\
Production Environment Setup & I & R & C \\
Production Deployment & C & R & I \\
User Training & R & C & I \\
\hline
\multicolumn{4}{|c|}{\textbf{Phase 5: Closure}} \\
\hline
Deliverable Acceptance & R & A & I \\
Knowledge Transfer & C & R & A \\
Lessons Learned & R & C & C \\
Administrative Closure & R & I & I \\
\hline
\end{tabular}
\end{table}

\textbf{Legend:} R = Responsible, A = Accountable, C = Consulted, I = Informed

\newpage

\section{Task 2: Resource Loading Analysis}

\subsection{Initial Resource Loading (Before Leveling)}

The initial project schedule revealed significant resource allocation issues during the development phase (Weeks 8-13).

\subsubsection{Week-by-Week Analysis}

\textbf{Weeks 1-7 (Sep 1 - Oct 19): Planning Phase}
\begin{itemize}
    \item Balanced workload across team
    \item Average utilization: 80-100\%
    \item No significant conflicts identified
\end{itemize}

\textbf{Weeks 8-13 (Oct 20 - Nov 30): Development Phase - CRITICAL OVER-ALLOCATION}
\begin{itemize}
    \item \textbf{Gulsher Khan:} 48 hours/week (120\% allocation)
    \begin{itemize}
        \item Backend API Development (Flask) - 14 days
        \item Frontend Development (HTML/CSS) - overlapping start
        \item Environment setup responsibilities
    \end{itemize}
    
    \item \textbf{Ahsan Faraz:} 48 hours/week (120\% allocation)
    \begin{itemize}
        \item LangGraph Agent Development (complex, 29 days)
        \item Agent workflow design
        \item LangGraph implementation
        \item Testing and validation
    \end{itemize}
    
    \item \textbf{Dawood Hussain:} 24 hours/week (60\% allocation)
    \begin{itemize}
        \item Under-utilized during critical development phase
        \item Available capacity not leveraged
    \end{itemize}
\end{itemize}

\textbf{Weeks 14-16 (Nov 16 - Nov 30): Testing \& Deployment}
\begin{itemize}
    \item Return to balanced allocation
    \item All team members at 80-100\%
\end{itemize}

\subsection{Initial Resource Loading Data}

Table~\ref{tab:initial-loading} presents the week-by-week resource allocation before leveling.

\begin{table}[H]
\centering
\caption{Initial Resource Loading (Before Leveling)}
\label{tab:initial-loading}
\small
\begin{tabular}{|c|c|c|c|c|c|c|}
\hline
\textbf{Week} & \textbf{Dawood} & \textbf{Gulsher} & \textbf{Ahsan} & \textbf{Total} & \textbf{Status} \\
\hline
1-7 & 32-40h & 32-40h & 32-40h & 96-120h & Balanced \\
\hline
\rowcolor{yellow!20}
8-9 & 32h (80\%) & 42h (105\%) & 42h (105\%) & 116h & Minor Over-alloc \\
\rowcolor{yellow!20}
10-11 & 36h (90\%) & 42h (105\%) & 40h (100\%) & 118h & Minor Over-alloc \\
\hline
12-13 & 36h (90\%) & 40h (100\%) & 40h (100\%) & 116h & Balanced \\
\hline
14-16 & 40h & 36-40h & 32-40h & 108-120h & Balanced \\
\hline
\end{tabular}
\end{table}

\textbf{Key Observations:}
\begin{itemize}
    \item Minor over-allocation (105\%) for Gulsher and Ahsan during Weeks 8-11
    \item Manageable workload with proper task scheduling
    \item Dawood maintains consistent 80-90\% utilization
    \item Overall team capacity well-managed
\end{itemize}

\newpage

\subsection{Resource Histograms (Initial)}

Figure~\ref{fig:initial-histograms} shows the initial resource loading for each team member, clearly highlighting the over-allocation issues during the development phase.

\begin{figure}[H]
    \centering
    \includegraphics[width=\textwidth]{initial_individual_histograms.png}
    \caption{Initial Resource Loading Histograms (Before Leveling)}
    \label{fig:initial-histograms}
\end{figure}

\textbf{Histogram Analysis:}
\begin{itemize}
    \item \textbf{Red bars:} Over-allocation (>40 hours/week)
    \item \textbf{Orange bars:} Under-allocation (<40 hours/week)
    \item \textbf{Blue bars:} Normal allocation (40 hours/week)
    \item \textbf{Green line:} Standard capacity (40 hours/week)
\end{itemize}

The histograms clearly show:
\begin{enumerate}
    \item Gulsher's workload peaks at 48 hours/week for 6 consecutive weeks
    \item Ahsan's workload peaks at 48 hours/week for 4 consecutive weeks
    \item Dawood's workload drops to 24 hours/week during the same period
\end{enumerate}

\newpage

\subsection{Project-Level Resource Usage (Initial)}

Figure~\ref{fig:project-comparison} shows the total team resource usage before and after leveling.

\begin{figure}[H]
    \centering
    \includegraphics[width=\textwidth]{project_level_comparison.png}
    \caption{Project-Level Resource Usage: Before vs After Leveling}
    \label{fig:project-comparison}
\end{figure}

\textbf{Before Leveling (Top Chart):}
\begin{itemize}
    \item Total team capacity: 120 hours/week (3 members × 40 hours)
    \item Weeks 8-13: Exactly at capacity limit (120 hours)
    \item No buffer for unexpected issues
    \item High risk of schedule delays
\end{itemize}

\textbf{After Leveling (Bottom Chart):}
\begin{itemize}
    \item Extended to 17 weeks (from 16 weeks)
    \item Peak utilization: 120 hours (Week 14)
    \item More sustainable distribution
    \item Buffer time available for quality assurance
\end{itemize}

\newpage

\section{Task 3: Resource Leveling}

\subsection{Identified Resource Conflicts}

\subsubsection{Conflict 1: Gulsher Khan (Technical Lead)}

\textbf{Period:} Weeks 8-11 (Oct 20 - Nov 16) \\
\textbf{Allocation:} 105\% (42 hours/week) \\
\textbf{Over-allocation:} 2 hours/week (minor)

\textbf{Conflicting Tasks:}
\begin{itemize}
    \item Backend API Development (Flask) - 14 days
    \item Frontend Development (HTML/CSS/Flask) - initial phase
    \item System integration coordination
\end{itemize}

\textbf{Impact:}
\begin{itemize}
    \item Minor workload pressure during peak development
    \item Manageable with proper task scheduling
    \item Minimal risk to quality or timeline
\end{itemize}

\subsubsection{Conflict 2: Ahsan Faraz (AI/ML Developer)}

\textbf{Period:} Weeks 8-9 (Oct 20 - Nov 2) \\
\textbf{Allocation:} 105\% (42 hours/week) \\
\textbf{Over-allocation:} 2 hours/week (minor)

\textbf{Conflicting Tasks:}
\begin{itemize}
    \item LangGraph Agent Development - initial intensive phase
    \item Agent workflow design
    \item LangGraph implementation setup
\end{itemize}

\textbf{Impact:}
\begin{itemize}
    \item Minor workload increase during setup phase
    \item Manageable with focused effort
    \item Quality maintained through proper planning
\end{itemize}

\subsubsection{Opportunity: Dawood Hussain (Project Manager)}

\textbf{Period:} Weeks 8-11 (Oct 20 - Nov 16) \\
\textbf{Allocation:} 80-90\% (32-36 hours/week) \\
\textbf{Status:} Well-balanced

\textbf{Opportunity:}
\begin{itemize}
    \item Consistent utilization throughout project
    \item Available for coordination and support
    \item Effective PM capacity management
\end{itemize}

\subsection{Leveling Strategy Applied}

\subsubsection{Strategy 1: Task Redistribution}

\textbf{1. Optimized Task Scheduling}
\begin{itemize}
    \item Staggered Backend and Frontend development start times
    \item Reduced overlap between major development activities
    \item Impact: Reduces Gulsher's peak from 105\% to 100\%
    \item Rationale: Better task sequencing maintains quality
\end{itemize}

\textbf{2. Balanced LangGraph Development}
\begin{itemize}
    \item Distributed intensive work across multiple weeks
    \item Balanced daily workload throughout development
    \item Impact: Reduces Ahsan's allocation from 105\% to 100\%
    \item Rationale: Sustainable pace for complex AI work
\end{itemize}

\textbf{3. Optimized PM Involvement}
\begin{itemize}
    \item Dawood maintains consistent 90\% utilization
    \item Active coordination during critical development phases
    \item Impact: Better capacity utilization (80\% → 90\%)
\end{itemize}

\newpage

\subsubsection{Strategy 2: Task Decomposition}

\begin{itemize}
    \item Broke down large tasks into smaller, manageable chunks
    \item Created clear handoff points for collaborative work
    \item Distributed sub-tasks based on availability
\end{itemize}

\subsubsection{Strategy 3: Buffer Management}

\begin{itemize}
    \item Used available project buffer (7 days)
    \item Extended critical activities rather than compress
    \item Prioritized quality over aggressive timeline
\end{itemize}

\subsection{Leveling Results}

\begin{table}[H]
\centering
\caption{Resource Utilization: Before vs After Leveling}
\label{tab:leveling-results}
\begin{tabular}{|l|c|c|c|}
\hline
\textbf{Team Member} & \textbf{Before} & \textbf{After} & \textbf{Improvement} \\
\hline
Gulsher Khan (Peak) & 105\% & 100\% & -5\% \\
Ahsan Faraz (Peak) & 105\% & 100\% & -5\% \\
Dawood Hussain (Avg) & 80\% & 90\% & +10\% \\
\hline
\textbf{Over-allocated Weeks} & 4 weeks & 0 weeks & \textbf{Eliminated} \\
\hline
\end{tabular}
\end{table}

\textbf{Schedule Impact:}
\begin{itemize}
    \item Project duration: 112 days (maintained)
    \item Completion: November 30, 2025
    \item Critical path maintained
    \item Resource leveling applied without extending deadline
\end{itemize}

\textbf{Quality \& Risk Benefits:}
\begin{itemize}
    \item Eliminated burnout risk
    \item Adequate time for complex LangGraph development
    \item More thorough integration testing (14 days vs 10)
    \item Reduced technical debt risk
    \item Better team morale and retention
\end{itemize}

\newpage

\subsection{Leveled Resource Histograms}

Figure~\ref{fig:leveled-histograms} shows the resource loading after applying leveling techniques.

\begin{figure}[H]
    \centering
    \includegraphics[width=\textwidth]{leveled_individual_histograms.png}
    \caption{Leveled Resource Loading Histograms (After Leveling)}
    \label{fig:leveled-histograms}
\end{figure}

\textbf{Key Improvements:}
\begin{itemize}
    \item All bars in green (balanced allocation)
    \item No over-allocation periods
    \item Smooth workload distribution
    \item Extended to 17 weeks to accommodate changes
\end{itemize}

\subsection{Stacked Resource Distribution}

Figure~\ref{fig:stacked-comparison} provides a stacked view of resource distribution before and after leveling.

\begin{figure}[H]
    \centering
    \includegraphics[width=\textwidth]{stacked_comparison.png}
    \caption{Stacked Resource Distribution: Before vs After}
    \label{fig:stacked-comparison}
\end{figure}

The stacked charts clearly show:
\begin{enumerate}
    \item \textbf{Before:} Weeks 8-13 reach exactly 120 hours (team capacity limit)
    \item \textbf{After:} More balanced distribution across all weeks
    \item \textbf{Extension:} One additional week added for sustainable completion
\end{enumerate}

\newpage

\section{Task 4: Updated Network Diagram \& Schedule}

\subsection{Schedule Adjustments}

\subsubsection{Major Changes}

\textbf{1. LangGraph Agent Development}
\begin{itemize}
    \item Duration: 24 → 29 days (+5 days)
    \item Sub-tasks extended proportionally:
    \begin{itemize}
        \item Agent Workflow Design: 6 → 7 days
        \item LangGraph Implementation: 9 → 12 days
        \item Testing \& Validation: 6 → 8 days
        \item Integration: 2 → 3 days
    \end{itemize}
\end{itemize}

\textbf{2. Frontend Development (HTML/CSS/Flask)}
\begin{itemize}
    \item Start date delayed: Oct 25 → Nov 6 (+12 days)
    \item Duration unchanged: 28 days
    \item Finish date: Nov 30 (instead of Nov 20)
\end{itemize}

\textbf{3. Integration Testing}
\begin{itemize}
    \item Duration: 10 → 14 days (+4 days)
    \item More thorough testing with balanced team
    \item Start: Dec 1 (delayed from Nov 21)
\end{itemize}

\textbf{4. All Subsequent Activities}
\begin{itemize}
    \item Shifted by 10-12 days
    \item Proportional delay maintained
    \item Dependencies preserved
\end{itemize}

\newpage

\subsection{Updated Critical Path}

The critical path has been extended but maintains its integrity:

\textbf{Critical Path Sequence (119 days):}
\begin{enumerate}
    \item Market Research (1.2.1) → Stakeholder ID (1.2.2)
    \item Business Case (1.2.4) → Project Authorization (M1)
    \item Requirements Gathering (1.3.1) → Requirements Approved (M2)
    \item System Architecture (1.3.2) → Risk Planning (1.3.6)
    \item Design Approved (M3) → Environment Setup (1.4.1)
    \item Backend API (1.4.2) → Frontend Dev (1.4.4)
    \item Integration Testing (1.4.5) → Development Complete (M4)
    \item Functional Testing (1.5.1) → UAT (1.5.2)
    \item Deployment (1.5.4) → Training (1.5.5) → Go Live (M5)
    \item Closure activities (1.6.1 → 1.6.2 → 1.6.3 → 1.6.4)
    \item Project Closed (M6)
\end{enumerate}

\textbf{Parallel Critical Path:}
\begin{itemize}
    \item M3 → LangGraph Agent (1.4.3) → Sub-tasks → Integration Testing (1.4.5)
\end{itemize}

\subsection{Updated Network Diagram}

Figure~\ref{fig:network-diagram} shows the updated Activity-on-Node (AON) network diagram with resource leveling changes highlighted.

\begin{figure}[H]
    \centering
    \includegraphics[width=\textwidth]{updated_network_diagram.png}
    \caption{Updated Network Diagram (After Resource Leveling)}
    \label{fig:network-diagram}
\end{figure}

\textbf{Color Coding:}
\begin{itemize}
    \item \textbf{Black:} Critical path tasks (TS=0)
    \item \textbf{Blue:} Resource-leveled tasks (extended/delayed)
    \item \textbf{Light Blue:} Non-critical tasks (TS>0)
    \item \textbf{White:} Milestones
\end{itemize}

\newpage

\subsection{Updated Work Breakdown Structure}

Figure~\ref{fig:wbs-diagram} presents the updated WBS with resource leveling changes.

\begin{figure}[H]
    \centering
    \includegraphics[width=\textwidth]{updated_wbs.drawio.png}
    \caption{Updated Work Breakdown Structure}
    \label{fig:wbs-diagram}
\end{figure}

\textbf{WBS Highlights:}
\begin{itemize}
    \item 6 major phases
    \item 40 work packages and milestones
    \item Clear responsibility assignments
    \item Resource-leveled tasks highlighted in blue
    \item Professional color scheme (black, blue, white)
\end{itemize}

\subsection{Updated Gantt Chart}

Figure~\ref{fig:gantt-chart} shows the updated project schedule in Gantt chart format.

\begin{figure}[H]
    \centering
    \includegraphics[width=\textwidth]{GanttChartUpdated.png}
    \caption{Updated Gantt Chart (After Resource Leveling)}
    \label{fig:gantt-chart}
\end{figure}

\textbf{Gantt Chart Features:}
\begin{itemize}
    \item Timeline: September 1 - November 30, 2025
    \item All dependencies shown
    \item Critical path highlighted
    \item Milestones marked
    \item Resource assignments indicated
\end{itemize}

\newpage

\subsection{Schedule Comparison}

Table~\ref{tab:schedule-comparison} compares the initial and updated schedules.

\begin{table}[H]
\centering
\caption{Schedule Comparison: Initial vs Updated}
\label{tab:schedule-comparison}
\begin{tabular}{|l|c|c|c|}
\hline
\textbf{Metric} & \textbf{Initial} & \textbf{Updated} & \textbf{Change} \\
\hline
Project Duration & 112 days & 112 days & No change \\
Start Date & Sep 1, 2025 & Sep 1, 2025 & No change \\
End Date & Nov 30, 2025 & Nov 30, 2025 & No change \\
Critical Path Length & 112 days & 112 days & No change \\
\hline
Gulsher Peak Alloc & 105\% & 100\% & -5\% \\
Ahsan Peak Alloc & 105\% & 100\% & -5\% \\
Dawood Avg Alloc & 80\% & 90\% & +10\% \\
\hline
Over-allocated Weeks & 4 weeks & 0 weeks & Eliminated \\
LangGraph Dev & 24 days & 24 days & No change \\
Frontend Start & Oct 25 & Oct 27 & +2 days \\
Integration Testing & 10 days & 10 days & No change \\
\hline
\end{tabular}
\end{table}

\subsection{Key Schedule Changes Summary}

\begin{enumerate}
    \item \textbf{Phase 1 (Initiation):} No changes - completed as planned
    \item \textbf{Phase 2 (Planning):} No changes - completed as planned
    \item \textbf{Phase 3 (Development):} Major changes
    \begin{itemize}
        \item LangGraph extended by 5 days
        \item Frontend delayed by 12 days
        \item Integration testing extended by 4 days
    \end{itemize}
    \item \textbf{Phase 4 (Testing \& Deployment):} Shifted by 10-12 days
    \item \textbf{Phase 5 (Closure):} Shifted by 9-11 days
\end{enumerate}

\newpage

\section{Benefits of Resource Leveling}

\subsection{Sustainable Workload}

\begin{itemize}
    \item No team member exceeds 100\% allocation
    \item Reduced burnout risk
    \item Improved work-life balance
    \item Better long-term productivity
\end{itemize}

\subsection{Improved Quality}

\begin{itemize}
    \item Adequate time for complex LangGraph development
    \item More thorough integration testing (14 days vs 10)
    \item Reduced technical debt
    \item Better code review opportunities
\end{itemize}

\subsection{Risk Mitigation}

\begin{itemize}
    \item Lower schedule risk due to realistic allocation
    \item Reduced dependency on over-worked individuals
    \item Better contingency for unexpected issues
    \item Improved team morale and retention
\end{itemize}

\subsection{Better Resource Utilization}

\begin{itemize}
    \item Dawood's capacity better utilized (70\% → 85\%)
    \item More balanced team collaboration
    \item Clearer task ownership and accountability
\end{itemize}

\subsection{Realistic Schedule}

\begin{itemize}
    \item Stakeholder expectations properly set
    \item Buffer time for quality assurance
    \item Flexibility for scope adjustments
    \item Achievable milestones
\end{itemize}

\newpage

\section{Risks and Mitigation}

\subsection{Risk 1: Extended Timeline}

\textbf{Impact:} Project completes 7 days later than originally planned

\textbf{Mitigation:}
\begin{itemize}
    \item Communicate early with stakeholders
    \item Emphasize quality benefits
    \item Use buffer time wisely
    \item Monitor progress weekly
\end{itemize}

\subsection{Risk 2: Tech Stack Change Impact}

\textbf{Impact:} Flask/LangGraph may have learning curve

\textbf{Mitigation:}
\begin{itemize}
    \item Allocated extra time in LangGraph development
    \item Team training during early phases
    \item Technical documentation emphasis
    \item Pair programming for knowledge transfer
\end{itemize}

\subsection{Risk 3: Integration Complexity}

\textbf{Impact:} Flask backend + LangGraph + HTML frontend integration

\textbf{Mitigation:}
\begin{itemize}
    \item Extended integration testing (14 days)
    \item Early integration checkpoints
    \item Continuous integration practices
    \item Dedicated integration task (1.4.3.4)
\end{itemize}

\newpage

\section{Recommendations}

\subsection{Maintain Leveled Schedule}

\begin{itemize}
    \item Do not compress timeline to meet original deadline
    \item Quality and team health are priorities
    \item Communicate benefits to stakeholders
\end{itemize}

\subsection{Monitor Resource Utilization Weekly}

\begin{itemize}
    \item Track actual hours vs planned
    \item Adjust allocations proactively
    \item Address emerging conflicts early
\end{itemize}

\subsection{Leverage Dawood's Increased Capacity}

\begin{itemize}
    \item More active involvement in testing
    \item Enhanced stakeholder communication
    \item Risk monitoring and mitigation
    \item Documentation and knowledge management
\end{itemize}

\subsection{Protect LangGraph Development Time}

\begin{itemize}
    \item Most complex and critical component
    \item Requires sustained focus
    \item Quality directly impacts project success
    \item Do not compress this timeline
\end{itemize}

\subsection{Plan for Contingencies}

\begin{itemize}
    \item 7-day extension provides some buffer
    \item Identify tasks that could be fast-tracked if needed
    \item Maintain risk register
    \item Regular status reviews
\end{itemize}

\newpage

\section{Conclusion}

Resource leveling has transformed the Mindful Eating Agent project schedule from an aggressive, risky timeline to a sustainable, achievable plan. While the project duration increased by 7 days (6\% extension), the benefits far outweigh the cost.

\subsection{Key Achievements}

\begin{enumerate}
    \item \textbf{Eliminated all resource over-allocations}
    \begin{itemize}
        \item Gulsher: 120\% → 100\% allocation
        \item Ahsan: 120\% → 100\% allocation
        \item 6 weeks of over-allocation eliminated
    \end{itemize}
    
    \item \textbf{Improved team utilization and balance}
    \begin{itemize}
        \item Dawood: 70\% → 85\% utilization
        \item More balanced workload distribution
        \item Better collaboration opportunities
    \end{itemize}
    
    \item \textbf{Reduced quality and schedule risks}
    \begin{itemize}
        \item Adequate time for complex LangGraph development
        \item Extended integration testing (14 days)
        \item Reduced technical debt risk
    \end{itemize}
    
    \item \textbf{Created realistic stakeholder expectations}
    \begin{itemize}
        \item Completion: November 30, 2025 (on schedule)
        \item Transparent communication of resource optimization
        \item Quality-focused approach
    \end{itemize}
    
    \item \textbf{Enabled sustainable development pace}
    \begin{itemize}
        \item No burnout risk
        \item Improved team morale
        \item Better long-term productivity
    \end{itemize}
\end{enumerate}

\newpage

\subsection{Success Factors}

The successful resource leveling was achieved through:

\begin{enumerate}
    \item \textbf{Comprehensive Analysis}
    \begin{itemize}
        \item Detailed week-by-week resource loading
        \item Clear identification of conflicts
        \item Data-driven decision making
    \end{itemize}
    
    \item \textbf{Strategic Task Redistribution}
    \begin{itemize}
        \item Frontend development delayed strategically
        \item LangGraph development extended appropriately
        \item PM capacity better utilized
    \end{itemize}
    
    \item \textbf{Effective Communication}
    \begin{itemize}
        \item Clear RACI matrix
        \item Transparent schedule changes
        \item Stakeholder buy-in
    \end{itemize}
    
    \item \textbf{Quality Focus}
    \begin{itemize}
        \item Prioritized quality over speed
        \item Extended testing phases
        \item Reduced technical debt
    \end{itemize}
\end{enumerate}

\subsection{Final Remarks}

The updated schedule, with completion on November 30, 2025, provides a solid foundation for successful project delivery with high quality and team satisfaction. The resource leveling optimization maintains the original deadline while ensuring sustainable workload, which will pay dividends in:

\begin{itemize}
    \item Higher quality deliverables
    \item Better team morale and retention
    \item Reduced post-deployment issues
    \item Sustainable development practices
    \item Improved stakeholder satisfaction
\end{itemize}

The Mindful Eating Agent project demonstrates strong project management practices with clear scheduling, well-defined dependencies, and effective resource management. The positive outcomes from resource leveling position the project for successful completion within the planned timeframe and budget.

\newpage

\section{Appendices}

\subsection{Appendix A: Files Delivered}

\begin{enumerate}
    \item \texttt{resource\_assignment\_matrix.csv} - RACI matrix
    \item \texttt{initial\_resource\_loading.csv} - Pre-leveling data
    \item \texttt{leveled\_resource\_loading.csv} - Post-leveling data
    \item \texttt{resource\_conflicts\_analysis.md} - Detailed conflict analysis
    \item \texttt{updated\_schedule.csv} - Complete schedule with changes
    \item \texttt{updated\_wbs.csv} - Updated work breakdown structure
    \item \texttt{gantt\_chart\_visio.csv} - Visio-compatible Gantt data
    \item \texttt{updated\_network\_diagram.drawio} - AON diagram (Draw.io)
    \item \texttt{updated\_wbs.drawio} - WBS diagram (Draw.io)
    \item \texttt{initial\_individual\_histograms.png} - Resource histograms (before)
    \item \texttt{leveled\_individual\_histograms.png} - Resource histograms (after)
    \item \texttt{project\_level\_comparison.png} - Before/after comparison
    \item \texttt{stacked\_comparison.png} - Stacked resource distribution
    \item \texttt{updated\_network\_diagram.png} - Network diagram image
    \item \texttt{updated\_wbs.drawio.png} - WBS diagram image
    \item \texttt{GanttChartUpdated.png} - Gantt chart image
\end{enumerate}

\subsection{Appendix B: Tools Used}

\begin{itemize}
    \item \textbf{Microsoft Excel} - Resource calculations and data analysis
    \item \textbf{Python (matplotlib, pandas)} - Histogram generation and visualization
    \item \textbf{Draw.io} - Network diagram and WBS creation
    \item \textbf{CSV} - Data exchange format
    \item \textbf{LaTeX} - Professional report documentation
\end{itemize}

\subsection{Appendix C: Color Scheme}

All diagrams use a professional color palette:

\begin{itemize}
    \item \textbf{Black (\#000000):} Critical path tasks
    \item \textbf{Blue (\#1976d2):} Resource-leveled tasks
    \item \textbf{Light Blue (\#e3f2fd):} Non-critical tasks
    \item \textbf{White (\#ffffff):} Milestones
    \item \textbf{Gray (\#f5f5f5):} Phase headers
\end{itemize}

\newpage

\subsection{Appendix D: References}

\begin{enumerate}
    \item Project Management Institute (PMI). (2021). \textit{A Guide to the Project Management Body of Knowledge (PMBOK® Guide)} – Seventh Edition.
    
    \item Project Management Institute (PMI). (2017). \textit{Practice Standard for Scheduling} – Third Edition.
    
    \item Kerzner, H. (2017). \textit{Project Management: A Systems Approach to Planning, Scheduling, and Controlling}. John Wiley \& Sons.
    
    \item Larson, E. W., \& Gray, C. F. (2021). \textit{Project Management: The Managerial Process}. McGraw-Hill Education.
    
    \item Assignment 02 - Work Breakdown Structure and Project Scope (Previous submission)
    
    \item Assignment 03 - Timeline Estimation, Network Analysis, and Cost Management (Previous submission)
\end{enumerate}

\subsection{Appendix E: Team Contributions}

\begin{table}[H]
\centering
\caption{Team Member Contributions to Assignment 04}
\begin{tabular}{|l|p{10cm}|}
\hline
\textbf{Member} & \textbf{Contributions} \\
\hline
Dawood Hussain & Resource assignment matrix, schedule analysis, report writing, stakeholder communication \\
\hline
Gulsher Khan & Network diagram creation, WBS updates, Gantt chart development, technical documentation \\
\hline
Ahsan Faraz & Resource loading analysis, histogram generation, conflict identification, data analysis \\
\hline
\end{tabular}
\end{table}

\vspace{2cm}

\begin{center}
\textbf{--- End of Report ---}
\end{center}

\end{document}
